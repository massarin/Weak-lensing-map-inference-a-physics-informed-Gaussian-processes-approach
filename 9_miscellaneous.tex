\chapter{Miscellaneous}
\section{Fourier Transform}
For a continuous and unbounded field, Fourier Transform (FT) can be applied:
\begin{equation}
    f(\bm{x}) = \int_\Omega d^p\bm{k} F(\bm{k}) e^{i2\pi\bm{k}\cdot\bm{x}}.
\end{equation}
\begin{align}
    \bm{\theta} &\equiv \bm{x}\\
    \bm{l} &\equiv 2\pi\bm{k}.
\end{align}
We obtain,
\begin{align}
    f(\bm{\theta}) &= \int_\Omega d^p \left(\frac{\bm{l}}{2\pi}\right) F(\bm{l}) e^{i\bm{l}\cdot\bm{\theta}}\\
    & = \frac{1}{(2\pi)^p}\int_\Omega d^p\bm{l} F(\bm{l}) e^{i\bm{l}\cdot\bm{\theta}}.
\end{align}
In fact, in the flat sky approximation we can treat $C(l)$ as a 2D power spectrum on a plane, which gives \cite{dodelson}:
\begin{equation}
    w(\theta)=\frac{1}{4\pi^2}\int_\Omega d^2\bm{l} C(l) e^{i\bm{l}\cdot\bm{\theta}} 
\end{equation}

\section{Full vs Flat}
Full sky
\begin{align}
    w(X,Y) &=\langle \hat{\delta}(X), \hat{\delta}(Y)\rangle \nonumber \\
    &= \sum_{ll'}\sum_{mm'} Y^m_l \overline{Y}^{m'}_{l'} \langle \hat{\delta}_{mm'}\hat{\delta}_{ll'} \rangle \nonumber \\
    &= \sum_{ll'}\sum_{mm'} Y^m_l \overline{Y}^{m'}_{l'}\delta_{mm'}\delta_{ll'}C_l \nonumber \\
    &= \sum_{l}C_l\sum_{m} Y^m_l \overline{Y}^{m}_{l} \nonumber \\
    &=
\end{align}

\begin{align}
    w(\mu) &=\sum_{l}\frac{2l+1}{4\pi}C_lP_l(\mu)\\
    \implies \int \mu w(\mu) P_{l'}(\mu) &= \sum_{l}\frac{2l+1}{4\pi}C_l\int \mu P_{l'}(\mu)P_l(\mu)=\frac{1}{2\pi}C_{l'} \nonumber \\
    C_l &= 2\pi \int \mu w(\mu) P_{l}(\mu)
    \label{eq:fullcltow}
\end{align}
Flat sky
\begin{align}    w(\theta) 
    &=FT^{-1}C(\bm l)\\
    &=\int \frac{d^2l}{(2\pi)^2} e^{i\bm l \cdot \bm \theta}C(\bm l) \nonumber\\
    &=\int \frac{dl}{(2\pi)^2} l C(l)\int d\phi e^{il\theta cos(\phi)} \nonumber\\
    &=\int \frac{dl}{2\pi} l C(l) J_0(l\theta)
\end{align}
Now to prove the equivalence between the flat sky and full sky relation, plug in the full sky Eq. \ref{eq:fullcltow},
\begin{align*}
    C_l &= 2\pi \int \mu \int \frac{dl'}{2\pi} l' C(l') J_0(l'\theta(\mu)) P_{l}(\mu) \nonumber\\
    &= \int dl' l' C(l') \int \mu J_0(l'\theta(\mu)) P_{l}(\mu) \nonumber
\end{align*}
for $l\gg1, J_0(l'\theta)\rightarrow P_{l'}(\mu)$
\begin{align}
    &\sim \int dl' l' C(l') \int \mu P_{l'}(\mu) P_{l}(\mu) \nonumber\\
    &\sim \int dl' l' C(l') \frac{\delta_{ll'}}{l} \nonumber\\
    &= C(l)
\end{align}

\section{Discrete Fourier Transform}
Moving to the discrete and bounded case, a Discrete Fuorier Transform (DFT) approach is needed:
\begin{equation}
    f(\bm{n}) = \frac{1}{N_1N_2...N_p}\sum_{\bm{p}=0}^{\bm{N}-1} F(\bm{p}) e^{i2\pi\bm{p}\cdot \left( \bm{n}\circ \bm{N}^{\circ -1}\right)}.
\end{equation}
\begin{align}
    \bm{\theta} &\equiv \bm{L}\circ \bm{n}\circ \bm{N}^{\circ -1}\\
    \bm{l} &\equiv 2\pi\bm{p}\circ\bm{L^{\circ-1}}
\end{align}
We obtain,
\begin{equation}
    f(\bm{\bm{\theta}}) = \frac{1}{N_1N_2...N_p}\sum_{\bm{l}=0}^{2\pi\bm{L^{\circ-1}}(\bm{N}-1)} F(\bm{l}) e^{i\bm{l}\cdot\bm{\theta}}
\end{equation}
where we have made use of the properties of the Hadamard Product $\circ$, a pair-wise product operation. We can additionally define
\begin{equation}
    \bm{l_{max}} \equiv 2\pi\bm{L^{\circ-1}}(\bm{N}-1) 
\end{equation}
to obtain
\begin{equation}
    f(\bm{\bm{\theta}}) = \frac{1}{N_1N_2...N_p}\sum_{\bm{l}=0}^{\bm{l_{max}}} F(\bm{l}) e^{i\bm{l}\cdot\bm{\theta}}
\end{equation}

Applying this general equation to our 2D flat sky case of a field defined on a square box of size \code{(L, L)} and pixels \code{(N, N)}.
\begin{align}
    \bm{\theta} &\equiv \bm{n}\frac{L}{N}\\
    \bm{l} &\equiv 2\pi\frac{\bm{p}}{L}\\
    l_{max} &\equiv 2\pi\frac{N-1}{L}
\end{align}
\begin{equation}
    w(\theta)=\frac{1}{N^2}\sum_{\bm{l}=0}^{l_{max}} C(l) e^{i\bm{l}\cdot\bm{\theta}}
    \label{eq:DFTcl}
\end{equation}