%%%%%%%%%%%%%%%%%%%%%%%%%%%%%%%%%
%%%%%%%%%%%% CHAPTER %%%%%%%%%%%%
%%%%%%%%%%%%%%%%%%%%%%%%%%%%%%%%%
\chapter{Conclusion}
We have hereby introduced the tool of Gaussian processes to the landscape of map inference of cosmological fields, in particular weak lensing convergence. We considered how the 2-point statistics of cosmological fields changes when we are dealing with bounded and discrete maps in \textit{Sec. }\ref{sec:weak lensing}. We discussed the realisation of Gaussian and lognormal fields in \textit{Sec. }\ref{sec:field generation}, and showed their ability to recover the 2-point statistics that they encode, in \textit{Sec. }\ref{sec:gaussian and lognormal fields}. We have included masking and noise to the data to simulate realistic maps in \textit{Sec. }\ref{sec:data simulation}. We have shown that it is possible to apply physical knowledge about the 2-point correlation function of a cosmological field in order to set up a Gaussian process able to produce a Gaussian realisation of such a field. We considered different set ups for the Gaussian process kernel in \textit{Sec. }\ref{sec:gaussian process kernel} and showed how they fare against one another in \textit{Sec. }\ref{sec:gaussian process priors}, ultimately proving empirically that the half-range FFT model is the best. In \textit{Sec. }\ref{sec:gaussian process map reconstruction} we present an application of Gaussian processes to a noiseless masked convergence map, in order to showcase its ability to reconstruct a heavily masked map. Finally we present our results for the cosmological parameters inference with GPs, conditioning on noisy and masked data. When running the inference model on one cosmological parameter we recover both parameters within two sigmas, $\sigma_8 = 0.776\pm0.015$ and $\Omega_m = 0.284\pm0.010$. For the two parameter inference we observe the well known banana-shaped degeneracy between $\sigma_8$ and $\Omega_m$, as well as recovering $0.762\pm0.028$ within two sigmas.

In future studies GPs could be tested on maps with larger grids. In order to achieve a resolution of $\sim 3$ arcmin with a map of size $(10^\circ,10^\circ)$, a $200\times200$ grid is needed. Too big for a GP. The bottleneck is given by the inversion of the kernel matrix, see \eqref{eq:conditioning}. Approximations of this operation could enable the use of GPs on larger grids. This can lead to the possibility of applying this method on current weak lensing catalogues and perhaps even full sky catalogues. Another pathway to explore are lognormal fields, as they do a much better job at simulating data than GRFs. Due to GPs being Gaussian, their associated likelihood is not suitable to treat lognormal fields. A modified likelihood could therefore unlock a correct application of GPs to lognormal fields.

\section*{Acknowledgments}
This project wouldn't have been possible without the author of the idea, Dr. Tilman Tröster. You guided me through my first real research experience and I am grateful. Thank you Veronika Oehl for always being there and for the very helpful discussions. I also express my gratitude to the Cosmology group at ETH, led by Prof. Alexandre Réfrégier. Hearing about my every progress every Monday morning for six months, couldn't have been easy, thank you. It has been an incredible experience, long and grinding, which I embarked upon with my friends and colleagues Tommaso and Pietro. Thank you for making these past few months memorable. Thanks to Guido van Rossum for giving us \code{Python}. Thank you Mia, for your unconditional support, \textit{you} keep me grounded. Non sarei qua senza di te mamma, grazie.

\bigskip

\noindent
\hfill \llap{\textit{Thank you, reader.}}