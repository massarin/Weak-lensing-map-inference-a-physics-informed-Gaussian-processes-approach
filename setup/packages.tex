%% See the TeXed file for more explanations

%% LaTeX Font encoding -- DO NOT CHANGE
\usepackage[OT1]{fontenc}

%% Babel provides support for languages.  'english' uses British
%% English hyphenation and text snippets like "Figure" and
%% "Theorem". Use the option 'ngerman' if your document is in German.
%% Use 'american' for American English.  Note that if you change this,
%% the next LaTeX run may show spurious errors.  Simply run it again.
%% If they persist, remove the .aux file and try again.
\usepackage[english]{babel}

%% Input encoding 'utf8'. In some cases you might need 'utf8x' for
%% extra symbols. Not all editors, especially on Windows, are UTF-8
%% capable, so you may want to use 'latin1' instead.
\usepackage[utf8]{inputenc}

%% This changes default fonts for both text and math mode to use Herman Zapfs
%% excellent Palatino font.  Do not change this.
%\usepackage[sc]{mathpazo}
\usepackage{lmodern}
%\usepackage[italic]{mathastext}
%\usepackage{newpxtext, newpxmath}
%\usepackage{newtx}

%% The AMS-LaTeX extensions for mathematical typesetting.  Do not
%% remove.
\usepackage{amsmath, bm, amssymb} %,,amsfonts,mathrsfs,

%% NTheorem is a reimplementation of the AMS Theorem package. This
%% will allow us to typeset theorems like examples, proofs and
%% similar.  Do not remove.
%% NOTE: Must be loaded AFTER amsmath, or the \qed placement will
%% break
\usepackage[amsmath,thmmarks]{ntheorem}

%% Newcommands
%\newtheorem*{definition*}{\textit{Definition.}}
%\newtheorem*{proof*}{\textit{Proof.}}
%\newtheorem{definition}{Definition}
\newcommand*{\code}{\lstinline[keepspaces=true,breaklines]}
\newcommand\x{0.28}

%% LaTeX' own graphics handling
\usepackage{graphicx}
\graphicspath{ {./images/} }
\usepackage{wrapfig}

%% Draw maps
\usepackage{tikz}
\usetikzlibrary{shapes.geometric}

%% Tables
\usepackage{capt-of}% or \usepackage{caption}
\usepackage{booktabs}
\usepackage{varwidth}

%% Valign
\usepackage[export]{adjustbox} % for valign option

%% We unfortunately need this for the Rules chapter.  Remove it
%% afterwards; or at least NEVER use its underlining features.
% \usepackage{soul}

%% declaration of originality.
\usepackage{pdfpages}

%% [NEED] This allows for additional typesetting tools in mathmode.
%% See its excellent documentation.
\usepackage{mathtools}

%% [NEED] Some extensions to tabulars and array environments.
\usepackage{array}

%% [OPT] Fancy package for source code listings.  The template text
%% needs it for some LaTeX snippets; remove/adapt the \lstset when you
%% remove the template content.
\usepackage{listings}
\lstset{language=TeX,basicstyle={\normalfont\ttfamily}}

%% [REC] Fancy character protrusion.  Must be loaded after all fonts.
%\usepackage[activate]{pdfcprot}

%% Make document internal hyperlinks wherever possible. (TOC, references)
%% This MUST be loaded after varioref, which is loaded in 'extrapackages'
%% above.  We just load it last to be safe.
\usepackage{hyperref} 
\hypersetup{colorlinks=true,
            linkcolor=blue,
            anchorcolor=black,
            citecolor=green,
            filecolor=cyan,
            menucolor=red,
            runcolor=cyan,
            urlcolor=cyan}

\usepackage[backref=true, backend=biber, style=phys]{biblatex} %Imports biblatex package
\emergencystretch=1em
\setcounter{biburllcpenalty}{7000}
\setcounter{biburlucpenalty}{8000}

\DefineBibliographyStrings{english}{%
  backrefpage = {page},% originally "cited on page"
  backrefpages = {pages},% originally "cited on pages"
}

\addbibresource{refs.bib} %Import the bibliography file
\usepackage{csquotes} % compatibility with biblatex

\usepackage{enumitem}
\setlist[itemize]{leftmargin=*}